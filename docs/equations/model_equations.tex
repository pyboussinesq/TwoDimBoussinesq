\documentclass[11pt]{article}
\usepackage{geometry}                
\geometry{a4paper,left=2.5cm,right=2.5cm,top=2.5cm,bottom=2.5cm}
\usepackage{natbib}
\usepackage{color}
\definecolor{mygreen}{RGB}{28,172,0} % color values Red, Green, Blue
\definecolor{mylilas}{RGB}{170,55,241}
\usepackage{epsfig}
\usepackage{amssymb,amsmath}
\usepackage{enumerate}
\usepackage{enumitem}
\usepackage[utf8]{inputenc}
\usepackage{hyperref}
\usepackage{mathtools}


\newcommand{\ssd}{\text{ssd}}
\newcommand{\sS}{\mathsf{S}}
\newcommand{\tot}{\text{tot}}

\usepackage[procnames]{listings}
\usepackage{color}

\begin{document}

\definecolor{keywords}{RGB}{255,0,90}
\definecolor{comments}{RGB}{0,0,113}
\definecolor{red}{RGB}{160,0,0}
\definecolor{green}{RGB}{0,150,0}

\lstset{language=Python, 
        basicstyle=\ttfamily\small, 
        keywordstyle=\color{keywords},
        commentstyle=\color{comments},
        stringstyle=\color{red},
        showstringspaces=false,
        identifierstyle=\color{green},
        procnamekeys={def,class}}
 

\input{symbols}

\section*{TwoDimBoussinesq}

This gives a very brief introduction to how the Two-Dimensional Boussinesq equations are derive.

\subsection*{3D Model Equations}

The three-dimensional Boussinesq equations is a simplified model from the Navier-Stokes equations.  It is valid in the limit where the density variations are small compared to the mean density
$$
\rho' \ll \overline{\rho}
$$
This is a very good approximation for ocean dynamics and sometimes appropriate for the atmosphere.  In this limit the density is approximated everywhere with a constant except in the buoyancy term.   A very general way to write it, as used in the MITgcm is,
\begin{align*}
\frac{D\vec u}{Dt} +  2 \vec \Omega \times \vec u 
+ \frac{g \rho}{\overline{\rho}} \hat k + \frac{1}{\overline{\rho}} \vec\nabla p 
& = \frac{1}{\overline{\rho}} \vec\nabla \cdot \vec \tau, \\
 \vec\nabla \cdot \vec u & = 0, \\
\frac{D S}{Dt}  & = 0, \\
\frac{D \theta}{Dt} &= \frac{1}{\overline{\rho} c_{pS}} \vec\nabla \cdot \mathcal{F}_\theta, \\
\rho & = \rho(\theta, S, z).
\end{align*}

\subsection*{2D Model Equations: Vorticity Approach}

There are many problems that can be studied in a two-dimensional context and are still interesting.  To study stratified flows we assume that the fields can vary in one horizontal direction, say $x$, and in the vertical, $z$  We allow for motion in the $y$ direction but no changes in that direction.  In the 2D context the problem can be set up using the pressure method (cite something here) or using the vorticity method.  Since TwoDimboussinesq focuses on the latter we focus on this approach.

Conservation of mass combined with the fact that no fields vary in the $y$ direction, allow us to define a two-dimensional streamfunction which then determines the velocity field, $(u,w)$,
$$
u = -\partial_z \psi, 
\quad \mbox{ and } \quad
w = \partial_x \psi.
$$
The momentum equations can be written in terms of the Jacobian,
\begin{align*}
\partial_t u   & =  J(\psi,  \partial_z \psi) + f v  +\nu \nabla^2 u  -  \frac{1}{{\rho}_0} \partial_x p, \\
\partial_t v  & =   - J(\psi, v)  -f u  +\nu \nabla^2 v , \\
\partial_t w   & =  - J(\psi, \partial_x \psi )  + b  +\nu \nabla^2 w  -  \frac{1}{{\rho}_0} \partial_z p.
\end{align*}

To form the $y$-component of vorticity, $\nabla^2 \psi = - \partial_z u + \partial_x w$ we compute the negative $z$ derivative of the $u$ eqn and add the $z$ derivative of the $w$ eqn,
$$
\partial_t \nabla^2 \psi = -J(\psi, \nabla^2 \psi) -  f v_z   + \partial_x b + \nu \nabla^2 \nabla^2 \psi
$$

Therefore the governing equations are,
\begin{align*}
\partial_t \nabla^2 \psi &= -J(\psi, \nabla^2 \psi)-  f v_z   +  \partial_x b + \nu \nabla^2 \nabla^2 \psi, \\
\partial_t v  & =   -J(\psi, v)  + f \partial_z \psi  +\nu \nabla^2 v , \\
\partial_t b & = -J(\psi, b) + \kappa \nabla^2 b.
\end{align*}

If we assume that there is a background linear stratification denoted with $\overline{b}(z) = N^2 z$ and decompose the buoyancy as
$$
b = \overline{b}(z) + b'.
$$
We substitute our decomposition into the equations and get,
\begin{align*}
\partial_t \nabla^2 \psi &= -J(\psi, \nabla^2 \psi)-  f v_z   +  \partial_x b + \nu \nabla^2 \nabla^2 \psi, \\
\partial_t v  & =   -J(\psi, v)  + f \partial_z \psi  +\nu \nabla^2 v , \\
\partial_t b & = -J(\psi, b) - N^2 \partial_x \psi + \kappa \nabla^2 b.
\end{align*}

\end{document}